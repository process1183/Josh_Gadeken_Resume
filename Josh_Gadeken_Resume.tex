% Josh Gadeken Resume
% Copyright (C) 2017 Josh Gadeken
\documentclass[letterpaper,10pt]{article}
\usepackage[includeheadfoot, top=1.5cm, bottom=1.5cm, left=1.5cm, right=1.5cm]{geometry}

\usepackage{fontspec}
\setmainfont[Ligatures=TeX]{Liberation Sans}

\usepackage[colorlinks=true,urlcolor=black]{hyperref}

\usepackage{fontawesome}

\usepackage{xcolor}

\usepackage{microtype}
\DisableLigatures{encoding = *, family = *}

% Josh Gadeken Resume
% Copyright (C) 2017 Josh Gadeken
\newcommand{\myphone}{REDACTED}
\newcommand{\myemail}{REDACTED}


\usepackage{lastpage}

\usepackage{fancyhdr}
\setlength{\headheight}{47.0pt}
\setlength{\footskip}{12pt}
\pagestyle{fancy}
\fancyhf{}
\setlength{\headsep}{20pt}
\renewcommand{\headrulewidth}{0pt}
\renewcommand{\footrulewidth}{0pt}
\fancyhead[C]{
    \Huge\textbf{Josh Gadeken} \\
    \vspace*{1ex}
    \normalsize
    \setlength{\tabcolsep}{4pt}
    \begin{tabular}{c|c|c|c}
        \faEnvelope \hspace*{1pt} \href{mailto:\myemail}{\myemail} & \faMobile \hspace*{1pt} \myphone & \faLinkedinSquare \hspace*{1pt} \href{https://www.linkedin.com/in/joshgadeken}{linkedin.com/in/joshgadeken} & \faGithub \hspace*{1pt} \href{https://github.com/process1183}{github.com/process1183}
    \end{tabular}
}
\fancyfoot[L]{\textcolor{gray}{Josh Gadeken R\'{e}sum\'{e}}}
\fancyfoot[R]{\textcolor{gray}{\thepage\ / \pageref*{LastPage}}}

\usepackage{sectsty}
\sectionfont{\mdseries\scshape\sectionrule{0pt}{0pt}{-1ex}{0.1pt}}

\usepackage{enumitem}
\setlist[itemize]{itemsep=-1pt}

\begin{document}
    \section*{Experience}
        \textbf{Aleph Objects, Inc.} (Loveland, Colorado) \hfill \textbf{July 2017 - May 2019} \\[1pt]
        \emph{Junior System Administrator}
        \begin{itemize}
            \item Maintained existing Debian workstations which involved diagnosing and correcting OS and application problems, replacing faulty hardware, and applying updates. OS installation for new workstations was automated with Debian preseed and postinstall scripts.
            \item Installed and configured an OpenNebula virtualization server to speed up internal services as well as reduce cost from hosting some of these internal services with cloud providers.
            \item Added a secondary Internet connection to a building's PfSense firewall. This secondary Internet connection was added to eliminate downtime caused by a failure on either connection.
            \item Set up OpenVPN on company workstations, laptops, and cell phones. This is used to allow employees to work and access internal services remotely.
            \item Set up and maintained company cell phones, which involved replacing the stock Android OS with LineageOS, and VoIP phones, which included configuring an Asterisk server.
        \end{itemize}
        % end AO
        \vspace*{2ex}
        \textbf{Hewlett-Packard / Hewlett Packard Enterprise} (Fort Collins, Colorado) \hfill \textbf{April 2015 - October 2016} \\[1pt]
        \emph{Systems Software Engineer}
        \begin{itemize}
            \item Worked on HPE's Debian derivative Linux distribution (HPE Linux), which is used as the base for several products, including HPE Helion and The Machine.
            \item As lead developer for the HPE Linux ISOs/installers, I maintained and added new features to the installer creation tools as well as building and testing new ISOs/installers. I also transitioned the proprietary installer creation source code to Debian's open source tools (simple-cdd) and automated the ISO/installer generation process.
            \item Contributed to Debian quality by submitting several bug reports with patches upstream to Debian. I tested my patches for these bugs on local QEMU virtual machines where I reproduced the issue, applied my patch, built the package, and then verified my solution to the bug.
            \item Developed Python and Bash tools for HPE Linux Debian repository creation and maintenance. These tools enabled automatic repository creation, and provided human and machine readable insights into the repositories.
            \item Developed automated tasks for CI/CD pipeline to build and validate packages, installers, and repositories. This pipeline is built with a proprietary HPE framework, Jenkins, and GitHub Enterprise. This pipeline enabled automatic scheduled repository and installer builds, as well as automatic package building and testing triggered by Git commits.
            \item Wrote and maintained documentation for tools, tests, and processes using DokuWiki. This enabled other coworkers and teams to find, use, and build on existing solutions, instead of having to create their own.
        \end{itemize}
        % end HPE
        \vspace*{2ex}
        \textbf{Active Website / Booj} (Lakewood, Colorado) \hfill \textbf{September 2010 - October 2013}
        \begin{itemize}
            \item Started as an intern, then promoted to Web Developer, and again promoted to Data Specialist.
            \item Wrote a multi-threaded Python framework to automatically download data in various formats from multiple clients and organize it into MySQL database for use on client websites.
            \item Improved media server performance by optimizing and upgrading media storage methods and by using Nagios to find and eliminate bottlenecks in media server Python code base. Amazon CloudFront CDN was later implemented as client site traffic increased.
            \item Set up new CentOS based Web and Media Server clusters to support increased traffic and additional client sites.
            \item Created and maintained features for front-end and back-end website functionality using PHP and Smarty Templates.
        \end{itemize}
        % end Booj
    % end experience section

    \newpage

    \section*{Skills}
        \begin{tabular}{lllll}
            Python (2 and 3) & Git      & Debian (server and desktop) & Untangle NG Firewall & Soldering / Desoldering \\
            Bash             & GitHub   & Ansible                     & Raspberry Pi         & Electronics Schematics \\
            C                & OpenSCAD & Software RAID (mdadm)       & Arduino              & Multimeter Use \\
            GNU Make         & \LaTeX   & libvirt                     & KerboScript (kOS)    & Electronics Prototyping \\
        \end{tabular}
    % end skills section

    \section*{Education}
        \begin{itemize}
            \item Front Range Community College
            \begin{itemize}
                \item Associate of Science: Fall 2019 - Present
            \end{itemize}
            \item Debian Packaging class by Tim Potter at HPE
            \begin{itemize}
                \item In this class I learned how Debian packages are structured and how to create them both manually and automatically using standard open source Debian tools. This also included best practices for Debian packaging and how to work with the Debian community.
            \end{itemize}
            \item Practical Python Programming class by David M. Beazley
            \begin{itemize}
                \item This course covered several Python concepts such as working with various types of data, modules, libraries, object oriented programming and classes, debugging, lambdas, and decorators.
            \end{itemize}
            \item Resurrection Christian School
            \begin{itemize}
                \item High School Diploma: 2009
            \end{itemize}
        \end{itemize}
    % end education section

    \section*{Personal Projects}
        \begin{description}[style=nextline]
            \item [Servers] My home server rack is comprised of several custom-built Linux systems. Among them is a firewall/router/UTM that runs Untangle NG and manages several home LANs, a QEMU virtualization server, and a backup server. Some of the virtual machines hosted on the QEMU virtualization server include a NAS, a VPN, a Debian APT cache, and a Home Assistant instance. The virtual machine server also uses Linux software RAID (mdadm) for storage.
            \item [Roomba RPi] I enhanced the functionality of my iRobot Roomba 690 by adding a Raspberry Pi Zero W, Inertial Measurement Unit (IMU), and Real-Time Clock (RTC) into the Roomba. The first of many planned features that I have completed is enabling wireless remote control of the  Roomba with a PlayStation 4 controller. Future plans include using this as a platform for the Robot Operating System (ROS) and improving the cleaning pattern by using the Roomba's existing sensors combined with the added IMU.
            \item [Hacktoberfest] Hacktoberfest is a celebration of open source software during the month of October run by DigitalOcean and GitHub. I completed the Hacktoberfest challenge for 2017 and 2018, which required submitting at least 5 pull requests to open source projects on GitHub.
            \item [Project Euler] Project Euler is a series of challenging mathematical and computer programming problems. I have solved several of these problems using both Python 3 and Rust.
        \end{description}
    % end personal projects section
\end{document}
